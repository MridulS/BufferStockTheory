\providecommand{\econtexRoot}{}
\renewcommand{\econtexRoot}{.}
\documentclass[../BufferStockTheory.tex]{subfiles}
\onlyinsubfile{% https://tex.stackexchange.com/questions/463699/proper-reference-numbers-with-subfiles
    \csname @ifpackageloaded\endcsname{xr-hyper}{%
      \externaldocument{\econtexRoot/BufferStockTheory}% xr-hyper in use; optional argument for url of main.pdf for hyperlinks
    }{%
      \externaldocument{main}% xr in use
    }%
    \renewcommand\labelprefix{}%
    % Initialize the counters via the labels belonging to the main document:
    \setcounter{equation}{\numexpr\getrefnumber{\labelprefix eq:AAgg}\relax}% eq:AAgg is the last numbered equation in the main text; start counting up from there
}

\begin{document}
\subsection{Convergence of $\vFunc_{t}$ in Euclidian Space}\label{sec:vEuclidian}

Boyd's theorem shows that $\TMap$ defines a contraction mapping
in a $\phiFunc$-bounded space. We now show that $\TMap$ also
defines a contraction mapping in Euclidian space.

Calling $\vFunc^{\ast}$ the unique fixed point of the operator $\TMap$, since $\vFunc^{\ast }(\mRat)=\TMap {\vFunc}^{\ast }(\mRat)$,
\begin{equation}
\left\Vert \vFunc_{T-n+1}-\vFunc^{\ast }\right\Vert _{\phiFunc }\leq \Shrinker
^{n-1}\left\Vert \vFunc_{T}-\vFunc^{\ast }\right\Vert _{\phiFunc }.
\end{equation}%
On the other hand, $\vFunc_{T}-\vFunc^{\ast }\in \mathcal{C}_{\phiFunc }\left( \mathscr{A},
\mathscr{B}\right) $ and $\MPC =\left\Vert \vFunc_{T}-\vFunc^{\ast }\right\Vert_{\phiFunc }<\infty $ because $\vFunc_{T}$ and $\vFunc^{\ast }$ are in $\mathcal{C}_{\phiFunc
}\left( \mathscr{A},\mathscr{B}\right) $. It follows that%
\begin{equation}
\left\vert \vFunc_{T-n+1}(\mRat)-\vFunc^{\ast }(\mRat)\right\vert \leq \MPC \Shrinker^{n-1}\left\vert \phiFunc(\mRat)\right\vert.
\end{equation}%
Then we obtain
\begin{equation}
\underset{n\rightarrow \infty }{\lim }\vFunc_{T-n+1}(\mRat)=\vFunc^{\ast }(\mRat).
\end{equation}

Since $\vFunc_{T}(\mRat)=\frac{\mRat^{1-\CRRA }}{1-\CRRA }$, $\vFunc_{T-1}(\mRat)\leq \frac{\left(\MaxMPC \mRat\right)^{1-\CRRA}}{1-\CRRA}<\vFunc_{T}(\mRat)$. On the other hand, $\vFunc_{T-1}\leq \vFunc_{T}$
means $\TMap\vFunc_{T-1}\leq \TMap\vFunc_{T}$, in other words, $\vFunc_{T-2}(\mRat)\leq \vFunc_{T-1}(\mRat)$.
Inductively one gets $\vFunc_{T-n}(\mRat)\geq \vFunc_{T-n-1}(\mRat)$. This means that $\left\{
\vFunc_{T-n+1}(\mRat)\right\} _{n=1}^{\infty }$ is a decreasing sequence,
bounded below by $\vFunc^{\ast}$.


\subsection{Convergence of $\cFunc_{t}$}
\label{subsec:cConverges}%\label{sec:cEuclidian}

Given the proof that the value functions converge, we now show the
pointwise convergence of consumption functions
$\left\{\cFunc_{T-n+1}(\mRat)\right\} _{n=1}^{\infty }$.

Consider any convergent subsequence $\left\{\cFunc_{T-n(i)}(\mRat)\right\}$ of $\left\{\cFunc_{T-n+1}(\mRat)\right\} _{n=1}^{\infty }$ converging to $c^*$. By the definition of $\cFunc_{T-n}(\mRat)$, we have 
\begin{equation}
    \uFunc(c_{T - n(i)}(\mRat)) + \DiscFac \Ex_{T - n(i)}[{\PGro}_{T-n(i)+1}^{1 - \CRRA}\vFunc_{T-n(i)+1}(\mRat)] \geq \uFunc(c_{T - n(i)}) + \DiscFac \Ex_{T - n(i)}[{\PGro}_{T-n(i)+1}^{1 - \CRRA}\vFunc_{T-n(i)+1}(\mRat)],
\end{equation}
for any $\cRat_{T - n(i)} \in [\MinMinMPC \mRat,\MaxMPC \mRat]$. Now letting $n(i)$ go to infinity, it follows that the left hand side converges to $\uFunc(\cRat^*) + \DiscFac \Ex_{t}[{\PGro}_{t}^{1 - \CRRA}\vFunc(\mRat)]$, and the right hand side converges to $\uFunc(\cRat_{T - n(i)}) + \DiscFac \Ex_{t}[{\PGro}_{t}^{1 - \CRRA}\vFunc(\mRat)]$. So the limit of the preceding inequality as $n(i)$ approaches infinity implies 
\begin{equation}
   \uFunc(\cRat^*) + \DiscFac \Ex_{t}[{\PGro}_{t + 1}^{1 - \CRRA}\vFunc(\mRat)] \geq \uFunc(\cRat_{T - n(i)}) + \DiscFac \Ex_{t}[{\PGro}_{t + 1}^{1 - \CRRA}\vFunc(\mRat)].
\end{equation}
Hence, $c^* \in \underset{\cRat_{T - n(i)} \in [\MinMinMPC \mRat,\MaxMPC \mRat]}{\arg \max }\left\{ \uFunc(\cRat_{T - n(i)}) + \DiscFac \Ex_{t}[{\PGro}_{t + 1}^{1 - \CRRA}\vFunc(\mRat)]\right\}$. By the uniqueness of $\cFunc(\mRat)$, $c^* = \cFunc(\mRat)$.

\begin{CDCPrivate}
We start by showing that
\begin{equation}
\cFunc(\mRat)=\underset{\cRat_{t}\in [\MinMinMPC \mRat,\MaxMPC \mRat]}{\arg \max }\left\{
\uFunc(c_{t})+\DiscFac \Ex_{t}\left[ {\PGro}_{t+1}^{1-\CRRA }\vFunc( {\mRat}_{t+1}) \right] \right\}
\end{equation}
is uniquely determined. We show this by contradiction. Suppose there
exist $c_{1}$ and $c_{2}$ that both attain the supremum for some $\mRat$,
with mean $\tilde{\cRat}=(c_{1}+c_{2})/2$. $c_{i}$ satisfies
\begin{equation}
\TMap{\vFunc}(\mRat)=\uFunc(c_{i})+\DiscFac \underbrace{\Ex_{t}\left[ {\PGro}_{t+1}
^{1-\CRRA }\vFunc(\mFunc_{t+1}( \mRat,\cRat_{i}) )
\right]}_{\equiv \mathfrak{v}}
\end{equation}
where $\mFunc_{t+1}(\mRat,\cRat_{i}) =(\mRat-\cRat_{i})\Rnorm_{t+1}+\tShkAll_{t+1}$ and
$i=1,2$. $\TMap{\vFunc}$ is concave for concave $\mathfrak{v}$. Since the space of
continuous and concave functions is closed, $\mathfrak{v}$ is also
concave and satisfies
\begin{equation}
\frac{1}{2}\sum_{i=1,2}\Ex_{t}\left[ {\PGro}_{t+1}^{1-\CRRA
}\vFunc( \mFunc_{t+1}( \mRat,\cRat_{i}) ) \right] \leq
\Ex_{t}\left[ {\PGro}_{t+1}^{1-\CRRA }\vFunc( \mFunc_{t+1}( \mRat,\tilde{\cRat}) ) \right].
\end{equation}
On the other hand, $\frac{1}{2}\left\{ \uFunc(c_{1})+\uFunc(c_{2})\right\} <\uFunc(
\tilde{\cRat}).$ Then one gets
\begin{equation}
\TMap \vFunc(\mRat)<\uFunc(\tilde{\cRat})+\DiscFac \Ex_{t}\left[ {\PGro}_{t+1}^{1-\CRRA
}\vFunc( \mFunc_{t+1}( \mRat,\tilde{\cRat}) ) \right].
\end{equation}

Since $\tilde{\cRat}$ is a feasible choice for $c_{i}$, the
LHS of this equation cannot be a maximum,  which contradicts the definition.

Using uniqueness of $\cFunc(\mRat)$ we can now show
\begin{equation}
\lim_{n\rightarrow \infty }\cFunc_{T-n+1}(\mRat) =\cFunc(\mRat).
\end{equation}
Suppose this does not hold for some $\mRat=\mRat^{*}$. In this case, $ \left\{
  \cFunc_{T-n+1}(\mRat^{*})\right\} _{n=1}^{\infty }$ has a subsequence $
\left\{ c_{T-n(i)}(\mRat^{*})\right\} _{i=1}^{\infty }$ that satisfies $
\lim_{i\rightarrow \infty }\cFunc_{T-n(i)}(\mRat^{*})=\cRat^{*}$ and $\cRat^{*}\neq
\cFunc(\mRat^{*})$. Now define $\cRat^{*} _{T-n+1}=\cFunc_{T-n+1}(\mRat^{*})$.  $\cRat^{*}>0$
because $\lim_{ i\rightarrow \infty }\vFunc_{T-n(i)+1}(\mRat^{*})\leq \lim_{
  i\rightarrow \infty }\uFunc(\cRat^{*}_{T-n(i)})$. Because $\aFunc(\mRat^{*})>0$ and
$\pShk \in [\ushort{\pShk},\bar{\pShk}]$ there exist $\{
\ushort{\mRat}_{+}^{*},\bar{\mRat}_{+}^{*}\} $ satisfying $0<
\ushort{\mRat}_{+}^{*}<\bar{\mRat}_{+}^{*}$ and ${\mRat}_{T-n+1}( \mRat^{*},\cRat^{*}
_{T-n+1}) \in \left[ \ushort{\mRat}_{+}^{*},\bar{\mRat}_{+}^{*}\right]$.
It follows that $\lim_{n\rightarrow \infty }\vFunc_{T-n+1}(\mRat)=\vFunc(\mRat)$~and the
convergence is uniform on~$\mRat\in
\left[\ushort{\mRat}_{+}^{*},\bar{\mRat}_{+}^{*}\right]$. (Uniform
convergence is obtained from Dini's theorem.\footnote{$\left[
    \text{\textbf{Dini's theorem}}\right] $ For a monotone sequence of
  continuous functions $\left\{ \vFunc_{n}(\mRat)\right\} _{n=1}^{\infty }$
  which is defined on a compact space and satisfies
  $\lim_{n\rightarrow\infty }\vFunc_{n}(\mRat)$ $=\vFunc(\mRat)$ where $\vFunc(\mRat)$ is
  continuous, convergence is uniform.}) Hence for any $\delta >0$,
there exists an $n_{1}$ such that
\begin{align*}
\DiscFac \Ex_{T-n}\left[ {\PGro}_{T-n+1}^{1-\CRRA }\left\vert
\vFunc_{T-n+1}( {\mRat}_{T-n+1}( \mRat^{*},\cRat^{*}
_{T-n+1}) ) -\vFunc( {\mRat}_{T-n+1}( \mRat^{*},
\cRat^{*}_{T-n+1}) ) \right\vert \right]  &<&\delta \notag
\end{align*}
for all $n\geq n_{1}$. It follows that if we define
\begin{equation}
\wFunc( \mRat^{*},z) =\uFunc(z)+\DiscFac \Ex_{T-n}\left[ {\PGro}_{T-n+1}^{1-\CRRA }\vFunc( {\mRat}_{T-n+1}( \mRat^{*}
,z) ) \right]
\end{equation}
then $\vFunc_{T-n}(\mRat^{*})$ satisfies
\begin{equation}
\lim_{n\rightarrow \infty }\left\vert \vFunc_{T-n}(\mRat^{*}
)-\wFunc( \mRat^{*},\cRat^{*}_{T-n+1}) \right\vert =0.
\label{Inequality1}
\end{equation}

On the other hand, there exists an $i_{1} \in \mathbb{N}$ such that
\begin{equation}
\left\vert \vFunc( \mFunc_{T-n( i) }( \mRat^{*},
\cRat^{*}_{T-n( i) }) ) -\vFunc( \mFunc
_{T-n( i) }( \mRat^{*},\cRat^{*}) )
\right\vert \leq \delta \text{ for all }i\geq i_{1}
\end{equation}
because $\vFunc$ is uniformly continuous on $[\ushort\mRat_{+}^{*},\bar\mRat_{+}^{*}]$. $\lim_{i\rightarrow \infty }\left\vert \cFunc_{T-n(i)}(\mRat^{*})-\cRat^{*}
\right\vert =0$ and

\begin{equation}
\left\vert \mFunc_{T-n( i) }( \mRat^{*},\cRat^{*}
_{T-n( i) }) -\mFunc_{T-n( i) }(
\mRat^{*},\cRat^{*}) \right\vert \leq \frac{\Rfree}{\PGro\ushort{\pShk}}
\left\vert \cRat^{*}_{T-n( i) }-\cRat^{*}\right\vert.
\end{equation}
This implies
\begin{equation}
\lim_{i\rightarrow \infty }\left\vert \wFunc(\mRat^{*},\cRat^{*}_{T-n(i)+1}) -\wFunc(\mRat^{*},\cRat^{*})
\right\vert =0  \label{Inequality2}.
\end{equation}
From \eqref{\localorexternallabel{Inequality1}} and \eqref{\localorexternallabel{Inequality2}}, we obtain $\lim_{i\rightarrow \infty }\vFunc_{T-n(i)}(\mRat^{*}
)=\wFunc( \mRat^{*},\cRat^{*}) $ and this implies $\wFunc(
\mRat^{*},\cRat^{*}) =\vFunc(\mRat^{*})$. This implies that $
\cFunc(\mRat)$ is not uniquely determined, which is a contradiction.

Thus, the consumption functions must converge.
\end{CDCPrivate}


\end{document}


