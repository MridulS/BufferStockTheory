\documentclass[../BufferStockTheory.tex]{subfiles}
\providecommand{\econtexRoot}{}
\renewcommand{\econtexRoot}{.}
 % Switch root to subdirectory 


\begin{document}

\section{The Limiting MPC's}
\label{sec:MPCLimits}
For $\mRat_{t}>0$ we can define $\eFunc_{t}(\mRat_{t})=\cFunc_{t}(\mRat_{t})/\mRat_{t}$
and $\aFunc_{t}(\mRat_{t})=\mRat_{t}-\cFunc_{t}(\mRat_{t})$
and the Euler equation \eqref{eq:scaledeuler} can be rewritten
\begin{eqnarray}
 \eFunc_{t}(\mRat_{t})^{-\CRRA} & = & \DiscFac \Rfree \Ex_{t}\left[\left(\eFunc_{t+1}({\mRat}_{t+1})\left(\frac{\overbrace{\Rfree \aFunc_{t}(\mRat_{t})+{\PGro}_{t+1}{\tShkAll}_{t+1}}^{=\mRat_{t+1}\PGro_{t+1}}}{\mRat_{t}}\right)\right)^{-\CRRA }\right] \label{eq:eFuncEuler}
\\ & = & \phantom{ + }\pNotZero \DiscFac \Rfree \mRat_{t}^{\CRRA} \Ex_{t}\left[ \left(\eFunc_{t+1}(\mRat_{t+1}) \mRat_{t+1}\PGro_{t+1}\right)^{-\CRRA} |~ \tShkAll_{t+1}>0 \right] \notag
\\ &  & + \pZero  \DiscFac \Rfree^{1-\CRRA} \Ex_{t}\left[\left(\eFunc_{t+1}(\mathcal{R}_{t+1}\aFunc_{t}(\mRat_{t}))\frac{\mRat_t-\cFunc_{t}(\mRat_{t})}{\mRat_{t}}\right)^{-\CRRA} |~ \tShkAll_{t+1} = 0 \right]  \label{eq:eRho}.
\end{eqnarray}



Consider the first conditional expectation in \eqref{eq:eRho},
recalling that if $\tShkAll_{t+1}>0$ then $\tShkAll_{t+1} \equiv
\tShkEmp_{t+1}/\pNotZero$.  Since $\lim_{m \downarrow 0}
\aFunc_{t}(\mRat) = 0$,
$\Ex_{t}[(\eFunc_{t+1}(\mRat_{t+1})\mRat_{t+1}\PGro_{t+1})^{-\CRRA}~|~\tShkAll_{t+1}>0]$
is contained within bounds defined by
$(\eFunc_{t+1}(\ushort{\tShkEmp}/\pNotZero) \PGro\ushort{\pShk}
\ushort{\tShkEmp}/\pNotZero)^{-\CRRA}$ and
$(\eFunc_{t+1}(\bar{\tShkEmp}/\pNotZero) \PGro\bar{\pShk}
\bar{\tShkEmp}/\pNotZero)^{-\CRRA}$ both of which are finite numbers,
implying that the whole term multiplied by $\pNotZero$ goes to zero as
$\mRat_{t}^{\CRRA}$ goes to zero.  As $\mRat_{t} \downarrow 0$ the
expectation in the other term goes to $\MaxMPC _{t+1}^{-\CRRA
}(1-\MaxMPC _{t})^{-\CRRA }.$ (This follows from the strict concavity
and differentiability of the consumption function.) It follows that
the limiting $\MaxMPC_{t}$ satisfies $\MaxMPC_{t}^{-\CRRA }=\DiscFac
\pZero\Rfree^{1-\CRRA }\MaxMPC_{t+1}^{-\CRRA }(1-\MaxMPC
_{t})^{-\CRRA }.$ Exponentiating by $\CRRA$, we can conclude that
\begin{eqnarray}
\MaxMPC_{t}&=&\pZero^{-1/\CRRA} (\DiscFac
\Rfree)^{-1/\CRRA}\Rfree(1-\MaxMPC _{t})\MaxMPC _{t+1} \notag
\\ \underbrace{\pZero^{1/\CRRA}\overbrace{{\Rfree}^{-1}(\DiscFac
    \Rfree)^{1/\CRRA}}^{\PatR}}_{\equiv \MinMPS}
\MaxMPC_{t}&=&(1-\MaxMPC _{t})\MaxMPC _{t+1} \label{eq:MinMPSDef}
\end{eqnarray}
which yields a useful recursive formula for the maximal marginal propensity to consume:
\begin{eqnarray}
  (\MinMPS \MaxMPC_{t})^{-1} & = & (1-\MaxMPC_{t})^{-1}\MaxMPC_{t+1}^{-1}  \notag
\\ \MaxMPC_{t}^{-1}(1-\MaxMPC_{t}) & = & \MinMPS \MaxMPC_{t+1}^{-1} \label{eq:muDiff} \notag
\\ \MaxMPC_{t}^{-1} & = & 1+\MinMPS \MaxMPC_{t+1}^{-1} \label{eq:MaxMPCInvApndx}.
\end{eqnarray}

As noted in the main text, we need the WRIC \eqref{eq:WRIC} for this to be a convergent sequence:
\begin{eqnarray}
  0 \leq & \pZero^{1/\CRRA} \PatR & < 1 \label{eq:WRICapndx},
\end{eqnarray}

Since $\MaxMPC_{T}=1$, iterating \eqref{eq:MaxMPCInvApndx} backward to
infinity (because we are interested in the limiting consumption function) we obtain:
\begin{eqnarray}
\lim_{n\rightarrow\infty}\MaxMPC_{T-n} & = &
\MaxMPC \equiv 1-\pZero^{1/\CRRA}\PatR  \label{eq:MaxMPCDef}
\end{eqnarray}
and we will therefore call $\MaxMPC$ the `limiting maximal MPC.'

The minimal MPC's are obtained by considering the case where
$\mRat_{t} \uparrow \infty$.  If the FHWC holds, then as
$\mRat_{t} \uparrow \infty$ the proportion of current and future
consumption that will be financed out of capital approaches 1.  Thus,
the terms involving $\tShkAll_{t+1}$ in \eqref{eq:eFuncEuler} can be
neglected, leading to a revised limiting Euler equation
\begin{eqnarray*}
 ({\mRat_{t}}\eFunc_{t}(\mRat_{t}))^{-\CRRA} & = & \DiscFac \Rfree \Ex_{t}\left[\left(\eFunc_{t+1}(\aFunc_{t}(\mRat_{t})\Rnorm_{t+1})\left({\Rfree \aFunc_{t}(\mRat_{t})}\right)\right)^{-\CRRA}\right]
\end{eqnarray*}
and we know from L'H\^opital's rule that $\lim_{\mRat_{t} \rightarrow \infty} \eFunc_{t}(\mRat_{t}) = \MinMPC_{t}$, and $\lim_{\mRat_{t} \rightarrow \infty} \eFunc_{t+1}(\aFunc_{t}(\mRat_{t})\Rnorm_{t+1}) = \MinMPC_{t+1}$ so a further limit of the Euler equation is
\begin{eqnarray*}
  (\mRat_{t}\MinMPC_{t})^{-\CRRA} & = & \DiscFac \Rfree \left(\MinMPC_{t+1} \Rfree (1-\MinMPC_{t})\mRat_{t}\right)^{-\CRRA}
\\ \underbrace{{\Rfree}^{-1}{\Pat}}_{\equiv \MaxMPS = (1-\ushort{\MPC})} \MinMPC_{t} & = & (1-\MinMPC_{t}) \MinMPC_{t+1}
\end{eqnarray*}
and the same sequence of derivations used above yields the conclusion
that if the RIC $0 \leq \PatR < 1$ holds, then a recursive formula for the
minimal marginal propensity to consume is given by
\begin{eqnarray}
 \MinMPC_{t}^{-1} & = & 1+\MinMPC_{t+1}^{-1}\MaxMPS  \label{eq:MinMPCInvApndx}
\end{eqnarray}
so that $\{\MinMPC_{T-n}^{-1}\}_{n=0}^{\infty}$ is also an increasing
convergent sequence, with $\MinMPC$ being the
`limiting minimal MPC.'  If the RIC does {\it not} hold, then $\lim_{n \rightarrow \infty} \MinMPC_{T-n}^{-1} = \infty$
and so the limiting MPC is $\MinMPC = 0.$


\end{document}
